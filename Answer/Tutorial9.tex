\begin{enumerate}
    \item A players pays \$1 to play a game where three fair dice are rolled. If three 6s are obtained, the player wins \$500. Otherwise the player wins nothing. What are the expected net winnings of this game? Would you want to play this game? Does your answer depend upon how many times you can play the game?
    
    $$\sigma = \sqrt{499^2\times\frac{1}{6^3} + (-1)^2(1-\frac{1}{6^3})} \approx 33.967304541$$
    $$E(X) = 499\times\frac{1}{6^3} + (-1)\left(1-\frac{1}{6^3}\right) = \frac{499}{216} - \frac{215}{216} = 71/54 = 1.31481481481$$

    The standard deviation is much greater than the expected value. This means that the risk factor is very high.

    Since the expected value does not account for the risk of ruin, it is likely that a player without enough capital will go bankrupt. A common approach is instead to calculate the expected rate of growth of wealth after the bet.

    Suppose we have \$$M$, then the expected rate of growth is:
    $$r = \left(\frac{M+499}{M}\right)^\frac{1}{216}\left(\frac{M-1}{M}\right)^\frac{215}{216}$$
    Only for $M >= 145$ is the rate of growth at least 1. Hence, for stability, I would only bet if I have at least \$145 regardless of how many bets I had.

    \item Suppose that the random variable X takes the values -2, 1, 4 and 6 with probability values 1/3, 1/6, 1/3 and 1/6 respectively.
    \begin{enumerate}
        \item Find the expectation of X.
        $$E(X) = -2\times \frac{1}{3} + 1\times\frac{1}{6} + 4\times\frac{1}{3} + 6\times\frac{1}{6} = 11/6$$
        \item Find the variance of X using the formula
        $$\n{Var}(X) = E((X - E(X))^2)$$
        $$\n{Var}(X) = (-2-11/6)^2\times \frac{1}{3} + (1-11/6)^2\times \frac{1}{6} + (4-11/6)^2\times \frac{1}{3} + (6-11/6)^2\times \frac{1}{6} = \frac{341}{36}$$
        \item Find the variance of X using the formula
        $$\n{Var}(X) = E(X^2) - (E(X))^2$$
        $$\n{Var}(X) = (-2)^2\times\frac{1}{3} + 1^2\times\frac{1}{6} + 4^2\times \frac{1}{3} + 6^2\times \frac{1}{6} - (11/6)^2 = \frac{341}{36}$$
    \end{enumerate}
    \item Suppose that you are organizing the game described in slide 7 of Lecture 9, where you charge players \$2 to roll two dice, and then you pay them the difference in the score.
    \begin{enumerate}
        \item What is the variance in your profit from each game? If you are playing a game in which you have positive expected winnings, would you prefer a small or a large variance in the winnings?\\
        The expected loss is:
        $$E(X) = 0\times\frac{1}{6}+1\times\frac{5}{18}+2\times\frac{2}{9}+3\times\frac{1}{6}+4\times\frac{1}{9}+5\times\frac{1}{18} = \frac{35}{18} \approx 1.944$$
        $$\n{Var}(X) = 0^2\times\frac{1}{6}+1^2\times\frac{5}{18}+2^2\times\frac{2}{9}+3^2\times\frac{1}{6}+4^2\times\frac{1}{9}+5^2\times\frac{1}{18}-\left(\frac{35}{18}\right)^2=\frac{665}{324}\approx 2.0524691358$$
        If the expected value is the same, I would prefer a small variance in the winnings.
        \item If you fix the dice so that each die has a probability of 0.2 of scoring a 3 and equal probability of 0.16 of scoring the other five numbers, do your expected winnings increase beyond 6 cents per game? Is it a surprise?
        \begin{align*}
            P(\n{diff}=0)=5(0.16^2)+0.2^2=0.168\\
            P(\n{diff}=1)=2(3(0.16\times0.16)+2(0.16\times0.2))=0.2816\\
            P(\n{diff}=2)=2(2(0.16\times0.2)+2(0.16\times0.16)) =0.2304\\
            P(\n{diff}=3)=2(2(0.16\times0.16)+(0.16\times0.2))=0.1664\\
            P(\n{diff}=4)=4(0.16\times0.16)=0.1024\\
            P(\n{diff}=5)=2(0.16\times0.16)=0.0512\\
        \end{align*}
        Hence the expected loss is:
        \begin{multline*}
            E(X')=0(0.168)+1(0.2816)+2(0.2304)+3(0.1664)+4(0.1024)+5(0.0512)\\
            =0+0.2816+0.4608+0.4992+0.4096+0.2560=1.9072
        \end{multline*}
        The expected earning is: $2-1.9072 = 0.0928$\\
        The earning before was already close to 6 cents, hence it is not really a surprise that the earning increases above 6 cents when there is a bias towards the middle.
    \end{enumerate}
    \item A random variable X has a probability density function $f(x) = A/\sqrt{x}$ for $3 \le x \le 4$.
    \begin{enumerate}
        \item What is the value of A?
        \begin{multline*}
            \int_{3}^{4}{f(x)\,dx}=\int_{3}^{4}{A/\sqrt{x}\,dx}=1\Rightarrow \left.2A\sqrt{x}\right|_{x=3}^{x=4}=1\\
            \Rightarrow 2A(2-\sqrt{3})=1\Rightarrow A = \frac{1}{2(2-\sqrt{3})} = \frac{2 + \sqrt{3}}{2}            
        \end{multline*}
        \item What is the cumulative distribution function of X?
        $$F(x)=\int_{3}^{x}{f(t)\,dt} = \int_{3}^{x}{A/\sqrt{t}\,dt}=1\Rightarrow \left.2A\sqrt{t}\right|_{t=3}^{t=x} = (2+\sqrt{3})(\sqrt{x}-\sqrt{3})$$
        \item What is the expected value of X?
        $$E(X)=\int_{3}^{4}{ax/\sqrt{x}\,dx}=\int_{3}^{4}{a\sqrt{x}\,dx}=\left.2A\sqrt{x}^3/3\right|_{x=3}^{x=4}\approx3.48803387171$$
        \item What is the standard deviation of X?
        $$\n{Var}(X) = \int_{3}^{4}{x^2f(x)\,dx}-E(X)^2 = \int_{3}^{4}{\frac{Ax^2}{\sqrt{x}}\,dx}-E(X)^2 = \left.\frac{2A}{5}\sqrt{x}^5\right|_{x=3}^{x=4} - E(X)^2 \approx 0.083361$$
        $$\sigma = \sqrt{\n{Var}(X)} \approx 0.288724732191$$
        \item What is the median of X?
        $$F(x)=0.5\Rightarrow(2+\sqrt{3})(\sqrt{x}-\sqrt{3})=0.5\Rightarrow \sqrt{x}=\frac{1}{2(2+\sqrt{3})}+\sqrt{3}=\frac{2+\sqrt{3}}{2}$$
        $$\Rightarrow x= \frac{7+4\sqrt{3}}{4}\approx 3.4820507$$
    \end{enumerate}
\end{enumerate}